\documentclass[10pt]{article}

\usepackage{multicol}
\usepackage{graphicx,amsmath,listings,hyperref,color,appendix,geometry}
\usepackage[english]{babel}
\usepackage{graphicx}
\usepackage{float}
\usepackage{mathtools}
\usepackage{fullpage}
\usepackage{listings,color}
\usepackage{xcolor}


%%\renewcommand{\bibsection}{}

\newfloat{listing}{H}{lop}
\floatname{listing}{Listing}

\renewcommand\lstlistingname{Appendices} % Change language of section name

\begin{document} 
\begin{figure}[!bh]
 	\begin{center}
 	 
 		\huge \title{Impacts of the HTTP/2 protocol for large scale web environments}
		\author{Martin Leucht, James Gratchoff \\
		Master SNE \\ University of Amsterdam} 
		\includegraphics{images/uva.jpeg}
	\maketitle 
		\label{sec:uva}
	\end{center}
\end{figure}
\pagenumbering{gobble}
\setlength{\columnsep}{2cm}
\def\columnseprulecolor{\color{blue}}
 
\newpage
\begin{abstract}
This paper compares version 1.1 and version 2 of the HTTP protocol. Both versions have been tested with TLS enabled. The main purpose is to show differences in terms of latency, header size, bandwidth utilisation and server performance between both protocol versions. In order to measure and compare the characteristics of the protocols, a benchmark setup has been developed. On server side different webserver implementaions were deployed in order to serve both protocol versions. HTTP clients are located around the world to represent different RTTs. The tool used for the protocol benchmarking is h2load. A script that calls h2load and is capable to provide our measurement requirements was developed. The results identify a clear improvement of the HTTP/2 protocol in terms of latency times and decreasing header sizes. Moreover the server performance will benefit from improvements that have been introduced in HTTP/2.
\end{abstract}

\newpage
\tableofcontents

\pagenumbering{arabic}
\newpage
%%\section*{Abstract}
\label{chap:intro}
\newpage
\section{Introduction}
\label{chap:intro}
HTTP/1.1 is present since 1999 and it is such a big protocol and has been reviewed so many time that the IETF has split the original RFC (2616) into six different ones (7230-7235) describing the whole protocol in details. After more than 15 years of use it was time for a change. All the tricks not to slow the HTTP/1.1 protocol can be forgotten as on February 18th 2015, the specification for the new HTTP protocol HTTP/2 (and HPACK), has been formally approved by the IESG and is on the way to become an RFC standard.  
The main focus during the development period of the successor of HTTP/1.1 was to improve the performance, and thus provide a better user experience. The performance improvement is mainly based on how the packets are sent over the wire within a HTTP/2 session. HTTP/2 data is sent in binary format and is based on a multiplexing mechanism that allows a single connection for parallelism. Concerning security, HTTP/2 will not make the use of TLS mandatory. However, leading browsers firms, Mozilla Firefox and Google Chrome have already mentioned that HTTP/2 will only be implemented over TLS. \\
There are already web client and server implementations that support the final HTTP/2 specification. This research is intending to show what are the impacts of implementing HTTP/2 with TLS compared to HTTPS in a large environment.
\section{Research Questions}
\label{chap:rq}

How do the new features of the  HTTP/2 protocol improve the performance for high frequently visited webpages/webserver?

\begin{itemize}
\item Are there specific usecases where the new features of the HTTP/2 protocol will provide major enhancements compared to the HTTP/1 protcol in large scale environments?
\item How can we measure such performance improvements that are proposed by the new protocol?
\item What are possible drawbacks that can occur for large web service providers when switching from HTTP/1.1 to HTTP/2 ?
\item What are predictable impacts that are related to changes in the infrastructure of Web service provider?
\end{itemize}
\section{Related work}
\label{relwork}
\section{Scope}
\label{scope}
This research will mainly focus on the major improvements of the HTTP/2 protocol compared to HTTP/1.1 and to investigate the general user experience when visiting a website (e.g. page loading time) and possible impacts on the infrastructure under high load circumstances. In order to have consistent results the same configuration will be used on different platforms implementing HTTP/2 and HTTP/1.1. 
\\
The conducted measurements are focussing on the new features of the protocol and are not intended to be a full benchmark of the protocol.

\input{sections/approach_and_method.tex}
\input{sections/approach_and_method/measurements.tex}
%%\input{sections/approach_and_method/topology.tex}
\subsection{HTTP Server and Clients}
\label{subsec:server_client}
HTTP clients that conduct measurements from long distances are deployed using Amazon t2.micro instances. T2 instances provide a baseline level of CPU performance that is comparable to 2 vCPUs with the ability to burst above the baseline level \cite{amazon-ts}. For local tests within the OS3 network, XEN VM are used. Each virtual machine had 2 vCPUs and 2 Gigabyte RAM assigned. The clients were located in different geographical areas, resulting in different RTT times between clients and server. On each client h2load was compiled and the wrapper script uploaded. Table \ref{table:locations} shows all clients and their corresponding average round trip times to the server.

\begin{table}[h]
	\centering
\begin{tabular}{ | c | c | }

\hline
\textbf{Location} & \textbf{RTT in ms}\\ \hline \hline
Tokyo/Asia &  280\\ \hline
North Carolina/North America &  150\\ \hline 
Frankfurt am Main/Europe &  7\\ \hline
Amsterdam/Europe (local) &  0.3\\

\hline
\end{tabular}
\caption{RTT (in ms) per location}
\label{table:locations}
\end{table}

The server that provides access for the HTTP clients is not virtualized. It has 8 Intel Xeon CPUs (1,87GHz) and 8GB RAM installed. The web server has a public IPv4 address and is connected to the public Internet via a 1 Gbit Ethernet interface. As HTTP/2 server nghttpd \cite{nghttp} version 0.76-DEV is used which listens on TCP port 8881. For HTTP/1.1 requests Apache2 \cite{apache2} in combination with nghttpx \cite{nghttpx} is used. The Apache2 configuration has been adjusted in order to allow the webserver to start a maximum number of 750 server worker processes. This adjustment was required because Apache starts multiple worker processes depending on the amount of requests, and we discovered that the default configuration was not sufficient for our setup. Nghttpx is a reverse proxy and accepts HTTP/2, SPDY and HTTP/1.1 over SSL/TLS on TCP port 8443 via its front-end. The most recent version of nghttpx (0.76-DEV) at the moment that research was done is used. The protocol to the back-end is HTTP/1.1. The usage of a reverse proxy enabled us to use our measurement tools without modification, although native HTTP/1.1 requests instead of using a reverse proxy would probably result in more accurate measurements. Since our time was very limited, we decided to go for the reverse proxy option.


\newpage
\section{Results analysis}
\label{results}
\input{sections/results/header.tex}
\input{sections/results/rtt.tex}
\subsection{Server Utilization}
\label{subsec:server_util}

During all measurements, CPU and network utilization on the server has been monitored and measured. The different results can be categorized into low latency and high latency measurements. For that reason only the server utilization for the local (low latency) and North America (high latency) measurements will be presented. The performance graphs for the server CPU utilization is shown in Figure \ref{fig:cpu}.  

\begin{figure}[H]
\centering
\includegraphics[scale=0.6,trim=0.0cm .0cm .0cm .0cm,clip]{images/cpu2.png}
\caption{Server CPU utilization for Local and North America measurements}
\label{fig:cpu}
\end{figure}

Table \ref{table:cpu} describes the coreresponding annotations that have been made to point out which measurement belongs to which measurement parameters. That table applies also for the next performance graphs since we always show exactly the same time range in all following graphs.

\begin{table}[h]
	\centering
\begin{tabular}{ | c | c | c | }

\hline
\textbf{Index} &\textbf{Protocol} &\textbf{Web Page Size}\\ \hline \hline
a & HTTP/2 & small.html (20kB)\\ \hline
b & HTTP/2 & medium.html (600kB)\\ \hline
c & HTTP/2 & large.html (1600kB)\\ \hline
d & HTTP/1.1 & small.html (20kB) \\ \hline
e & HTTP/1.1 & medium.html (600kB) \\ \hline
f & HTTP/1.1 & large.html (1600kB)\\ \hline 
g & HTTP/2 & small.html (20kB)\\ \hline
h & HTTP/2 & medium.html (600kB)\\ \hline
i & HTTP/2 & large.html (1600kB)\\ \hline
j & HTTP/1.1 & small.html (20kB) \\ \hline
k & HTTP/1.1 & medium.html (600kB) \\ \hline
l & HTTP/1.1 & large.html (1600kB)\\  
\hline
\end{tabular}
\caption{Different Measurement parameters}
\label{table:cpu}
\end{table} 
 
The attentive reader will now  notice that a medium.html web page shows up in Table \ref{table:cpu}. Indeed we created three different types of web pages. During the analysis of the data we realised that there is almost no significant difference in all measurements between the small size page and the medium size pages. For that reason we did not consider the measurements for the medium.html page and only focussed on the other two categories (large.html/medium.html). However, in the performance graphs of the server the measurement for the medium size web pages are included.
\\ 
The left part of the graph shows the CPU utilization for local measurements (low latency) on the server and is annotated with the letters \textit{a} to \textit{f}. A sudden increase up to 60\% from the beginning on staying steady until the measurements are finished is remarkable. That correlates to the request rates graph for local measurements and nicely presents a high CPU utilization caused by the high number of requests towards the server. The measurements from \textit{a} to \textit{c} show HTTP/1.1 measurements and from \textit{d} to \textit{f} HTTP/2 measurements. A significant difference regarding CPU utilization is not noticeable. Starting from 6pm until 4am (annotated by \textit{g} to \textit{l}) measurements over a high latency link (North America) were performed. We see a constant growing of the CPU graph until the maximum number of clients of 750 for each test is reached. A nearly identical characteristic shows the bandwidth utilization graph. That correlates to the request per second graph as well and reflects a growing load caused by constantly increasing number of requests. A significant difference between HTTP/1.1 and HTTP/2 is also not noticeable for the high latency measurements. Furthermore a different load characteristic for changing sizes of web pages is not measurable for both protocol variants. 
\\
The traffic graph (Figure \ref{fig:network} that represents the bandwidth utilization of the external interface of the server shows a similar characteristic compared to the CPU utilization graph (Figure \ref{fig:cpu}). One difference is the increased amount of traffic that can be explained by the increased amount of data for larger web pages. 

\begin{figure}[H]
\centering
\includegraphics[scale=0.6,trim=0.0cm .0cm .0cm .0cm,clip]{images/network.png}
\caption{Server Bandwidth utilization for Local and North America measurements}
\label{fig:network}
\end{figure}

It is worth to mention that for all HTTP/1.1 measurements a lot of TCP TIME WAIT connections were discovered on the server. TCP TIME WAITS occur when the endpoint (server) blocks a current connection before it can close it due to some missing packets for that particular session. That happens because the server has much more TCP connections to manage for HTTP/1.1 compared to HTTP/2 and shows clearly the benefit of using the HTTP/2 protocol. 
The graph representing the TCP socket states on the server is shown in Figure \ref{fig:sockets}.

\begin{figure}[H]
\centering
\includegraphics[scale=0.6,trim=0.0cm .0cm .0cm .0cm,clip]{images/sockets.png}
\caption{Server sockets for Local and North America measurements}
\label{fig:sockets}
\end{figure}
\newpage
\section{Conclusions}
\label{conclusion}

The research has identified differences between the HTTP/1.1 and HTTP/2 in terms of response and request time, amount of header data produced, possible request rates and impacts on the server utilization. A full benchmark setup has been developed consisting of several HTTP clients in different geographically locations and a server instance that delivers responses for both HTTP protocol versions. Moreover webpages of different sizes have been created to simulate "real-world" scenarios. In order to have accurate measurement results a benchmarking script \ref{sec:appendices} that makes use of h2load \cite{h2load} was deployed. 
\\
The measurements reflect the performance improvements a client will experience in the near future by using the new HTTP/2 protocol. Those improvements are mainly based on header compression mechanisms and more effective usage of TCP connections. Particulary high latency connections will benefit from the multiplexing mechanism that was introduced in HTTP/2. HTTP/2 will result in significant less TCP connection establishments (TCP three-way handshakes) on server side and thus it will save a lot of costly packet round trip times that would be required to fetch the same web page by using HTTP/1.1.
\\
It is important to consider that all measurements were performed by using clients running on the  Amazon EC2 infrastructure. Thus an assured allocation of the VM resources is not fully guaranteed. That is the drawback of a shared infrastructure and can have impacts on the measurements. Furthermore network pathes towards the webserver can change seen from the Amazon infrastructure or other events can happen that might impact round trip times.  
\\
A significant difference between HTTP/1.1 and HTTP/2 in terms of network load or CPU utilization was not detectable on the server. The main difference that has been discovered on server side is the TCP connection handling. It is faster and more effective when HTTP/2 is used, especially significantly less TCP TIME WAITS sockets were discovered on the server while performing HTTP/2 measurements.
\\
Large web service providers need to consider some technical implementation details when switching to the new protocol. First, so called deep inspection packet filters need to be adjusted in order to let HTTP/2 packets to pass through. Those devices cannot longer inspect HTTP/2 traffic on application level, like it was used to be with encrypted or unencrypted plain text HTTP/1.1 connections. The protocol is now binary and a simple telnet connection to a web server in order to conduct an HTTP GET request manually will not be possible with HTTP/2 anymore. Furhermore "hacks", like Sharding or Sprinting which were introduced in HTTP/1.1 to solve performance issues of the protocol are no longer necessary. In fact those HTTP/1.1 workarounds will lead to performance loss rather than to performance gain, if combined with HTTP/2.    


\newpage
\section{Further Work}
\label{furtherwork}
\newpage
\section*{}
\addcontentsline{toc}{section}{References}
\label{chap:references}
\begin{thebibliography}{99}
\bibitem{RFC723x}
RFC7230 Fielding, R., Ed., and J. Reschke, Ed., "Hypertext Transfer Protocol (HTTP/1.1): Message Syntax and Routing", RFC 7230, June 2014, Available at:http://www.rfc-editor.org/info/rfc7230  [Accessed 26 Mar. 2015]. \\
RFC7231 Fielding, R., Ed., and J. Reschke, Ed., "Hypertext Transfer Protocol (HTTP/1.1): Semantics and Content", RFC 7231, June 2014, Available at: http://www.rfc-editor.org/info/rfc7231 [Accessed 26 Mar. 2015]. \\
RFC7232 Fielding, R., Ed., and J. Reschke, Ed., "Hypertext Transfer Protocol (HTTP/1.1): Conditional Requests", RFC 7232, June 2014, Available at: http://www.rfc-editor.org/info/rfc7232 [Accessed 26 Mar. 2015]. \\ 
RFC7233 Fielding, R., Ed., Lafon, Y., Ed., and J. Reschke, Ed., "Hypertext Transfer Protocol (HTTP/1.1): Range Requests", RFC 7233, June 2014, Available at: http://www.rfc-editor.org/info/rfc7233 [Accessed 26 Mar. 2015]. \\
RFC7234 Fielding, R., Ed., Nottingham, M., Ed., and J. Reschke, Ed., "Hypertext Transfer Protocol (HTTP/1.1): Caching", RFC 7234, June 2014, Available at: http://www.rfc-editor.org/info/rfc7234 [Accessed 26 Mar. 2015]. \\
RFC7235 Fielding, R., Ed., and J. Reschke, Ed., "Hypertext Transfer Protocol (HTTP/1.1): Authentication", RFC 7235, June 2014, Available at: http://www.rfc-editor.org/info/rfc7235 [Accessed 26 Mar. 2015].
\bibitem{http2}
Hypertext Transfer Protocol version 2 (draft-ietf-httpbis-http2-17) (2015). M. Belshe, Twist R. Peon [online] IETF, Available at:
https://datatracker.ietf.org/doc/draft-ietf-httpbis-http2/?include\_text=1 [Accessed 26 Mar. 2015].
 \bibitem{hpack}
HPACK - Header Compression for HTTP/2 draft-ietf-httpbis-header-compression-07 (2015). R. Peon,  H. Ruellan [online] IETF, Available at: http://tools.ietf.org/html/draft-ietf-httpbis-header-compression-07 [Accessed 26 Mar. 2015].
\bibitem{stenberg}
Stenberg, D. (2015). http2 explained - The HTTP/2 book. [online] Daniel.haxx.se. Available at: http://daniel.haxx.se/http2/ [Accessed 27 Mar. 2015].
\bibitem{httpbis}
Tools.ietf.org, (2015). Httpbis Status Pages. [online] Available at: https://tools.ietf.org/wg/httpbis/ [Accessed 27 Mar. 2015].
\bibitem{spdy}
Google Developers, (2015). SPDY. [online] Available at: https://developers.google.com/speed/spdy/ [Accessed 27 Mar. 2015].
\bibitem{raymond}
Raymond, E. (2015). The Importance of Being Textual. [online] Catb.org. Available at: http://www.catb.org/esr/writings/taoup/html/ch05s01.html [Accessed 27 Mar. 2015].
\bibitem{curlhttp2}
Curl.haxx.se, (2015). cURL - README.http2. [online] Available at: http://curl.haxx.se/dev/readme-http2.html [Accessed 27 Mar. 2015].
\bibitem{wiresharkhttp2}
Wiki.wireshark.org, (2015). HTTP2 - The Wireshark Wiki. [online] Available at: https://wiki.wireshark.org/HTTP2 [Accessed 27 Mar. 2015].
\bibitem{breach}
Breachattack.com, (2015). BREACH ATTACK. [online] Available at: http://breachattack.com/ [Accessed 27 Mar. 2015].
\bibitem{crime}
Blackhat, (2013). A Perfect CRIME?. [online] Available at: https://media.blackhat.com/eu-13/briefings/Beery/bh-eu-13-a-perfect-crime-beery-wp.pdf [Accessed 27 Mar. 2015].
\bibitem{google2x}
A 2x Faster Web. (2009). [online] Chromium Blog. Available at: http://blog.chromium.org/2009/11/2x-faster-web.html [Accessed 20 Feb. 2015].
\bibitem{servy}
Servy, H. (2015). Evaluating the Performance of SPDY-enabled Web Servers. [online] Neotys.com. Available at: http://www.neotys.com/blog/performance-of-spdy-enabled-web-servers/ [Accessed 20 Feb. 2015].
\bibitem{podiatry}
Podiatry, G. (2015). Guy's Pod » Blog Archive » Not as SPDY as You Thought. [online] Guypo.com. Available at: http://www.guypo.com/not-as-spdy-as-you-thought/ [Accessed 20 Feb. 2015].
\bibitem{wang}
Wang et al. (2014). How Speedy is SPDY?, 11th USENIX Symposium on Networked Systems Design and Implementation (NSDI ’14). [online] usenix.org. Available at:
https://www.usenix.org/system/files/conference/nsdi14/nsdi14-paper-wang\_xiao\_sophia.pdf [Accessed 20 Feb. 2015].
\bibitem{amazon} Amazon Elastic Compute Cloud (Amazon EC2) (2015). Available at: http://aws.amazon.com/ec2/ [Accessed 20 Feb. 2015].
\bibitem{http2-imp}  HTTP/2 client/server implemenations (2015). Available at: https://github.com/http2/http2-spec/wiki/Implementations [Accessed 20 Feb. 2015].
\bibitem{h2c-14} Hypertext Transfer Protocol version 2 draft-ietf-httpbis-http2-14 (2014). Available at: https://tools.ietf.org/html/draft-ietf-httpbis-http2-14 [Accessed 4. March 2015]
\bibitem{nghttp} HTTP/2 experimental server (2015). Available at: https://nghttp2.org/documentation/nghttpd.1.html [Accessed 4. March 2015]
\bibitem{apache2} The Apache HTTP Server Project (2015). Available at: http://httpd.apache.org/ [Accessed 4. March 2015]
\bibitem{h2load} Benchmarking tool for HTTP/2 and SPDY server (2015). Available at: https://nghttp2.org/documentation/h2load.1.html [Accessed 4. March 2015]
\bibitem{httparchive} Httparchive.org, (2015). HTTP Archive - Trends. [online] Available at: http://httparchive.org/trends.php [Accessed 11 Mar. 2015].
\bibitem{nginx}
Garrett, O. (2015). How NGINX Plans to Support HTTP/2 - NGINX. [online] NGINX. Available at: http://nginx.com/blog/how-nginx-plans-to-support-http2/ [Accessed 27 Mar. 2015].
\bibitem{amazon-ts} amazon.com, (2015). Amazon - T2 Instances. [online] Available at: http://docs.aws.amazon.com/AWSEC2/latest/UserGuide/t2-instances.html [Accessed 11 Mar. 2015].
\bibitem{nghttpx} nghttp2.org, nghttpx a HTTP/1.1 HTTP/2 reverse proxy. [online] Available at: https://nghttp2.org/documentation/nghttpx.1.html [Accessed 1 April 2015].
\end{thebibliography}
\newpage
\section*{Appendices}
\addcontentsline{toc}{section}{Appendices}
\label{appendices}

\lstset{ % General setup for the package
	language=Perl,
	basicstyle=\small\sffamily,
	numbers=left,
 	numberstyle=\tiny,
	frame=tb,
	tabsize=4,
	columns=fixed,
	showstringspaces=false,
	showtabs=false,
	keepspaces,
	commentstyle=\color{gray},
	keywordstyle=\color{blue}
}

\begin{lstlisting}
#!/usr/bin/perl 
# Author Martin Leucht 2015 <martin.leucht@os3.nl>
#
# Benchmarking tool for HTTP/2 servers/reverse proxies
# based on h2load 
# https://nghttp2.org/documentation/h2load.1.html
#
# Usage: ./measure.pl [URL-list-file]
# Output is written to an CSV file

use List::Util qw(sum);
use strict;
use warnings;

my $file_urls = $ARGV[0];
my $runs=10;
my $stepsize_clients=2;
my $parallel_clients = 750;
my $bin_v2 = '/usr/local/bin/h2load';
my $start_datestring = gmtime();
my $time = time;
my $filename = "report_http_${file_urls}_${time}.csv";

sub check{
	my %units = ('1000' => 's','0.001' => 'us','1' => 'ms',);
	my $value = shift;
	my $unit = shift;
	
	foreach my $key (keys %units) {
		if ($units{$key} eq $unit) {
			our $ret = $key * $value;
			}
		}		
return our $ret;
}

sub average {
	my $size = @_;
	
	if ($size == 0) {
		print "NA";
		}
	else {return sum(@_)/@_;}
}


open(my $fh, '>', $filename) or die "Could not open file '$filename' $!";
print $fh "$start_datestring\n";
print $fh "number_of_requests,total_time_test,req_per_sec_v2,speed,traffic_total,traffic_header,
		   traffic_data,min_request,max_request,mean_request,sd_request,sd_percent_request,
		   succeeded_requests,succeeded_failed\n";
close $fh;



for (my $n=1; $n <= $parallel_clients; $n=$n+$stepsize_clients) {  
	my @finished_total_time_v2= ();
    my @finished_req_per_sec_v2 = ();
    my @finished_speed_v2 = ();
    my @finished_traffic_total_v2= ();
    my @finished_traffic_header_v2 = ();
    my @finished_traffic_data_v2 =();
    my @finished_min_request_v2 =();
    my @finished_max_request_v2 =();
    my @finished_mean_request_v2 =();
    my @finished_sd_request_v2 =();
    my @finished_sd_percent_request_v2 =();
    my @finished_succeeded =();
    my @finished_failed =();



	for (my $i=1; $i <= $runs; $i++) {
		
		my @command = `$bin_v2 -n $n -c $n --input-file=$file_urls --max-concurrent-streams=auto`;				
	
		foreach my $line (@command) {

			if ( $line =~ m/^finished.*?([0-9]*\.[0-9]*)(\w{1,2}).*?(\d+)\s.*?(\d+\.\d+).*/ ) {
				
				my $val = check($1,$2);	
				push(@finished_total_time_v2, $val);
				push(@finished_req_per_sec_v2, $3);
				push(@finished_speed_v2, $4);
				}
			elsif ( $line =~ m/^traffic.*?(\d+).*?(\d+).*?(\d+).*/ ) {	
				push(@finished_traffic_total_v2, $1);
                push(@finished_traffic_header_v2, $2);
                push(@finished_traffic_data_v2, $3);
				}
				
			elsif ( $line =~ m/^requests.*?(\d+)\s(?=succeeded).*?(\d+)\s(?=failed)/ ) {
				push(@finished_succeeded, $1);
				push(@finished_failed, $2);
				}	
					
			elsif ( $line =~ m/^time.*?(\d+\.*\d*)(\w{1,2}).*?(\d+\.*\d*)(\w{1,2}).*?(\d+\.*\d*)(\w{1,2}).*?(\d+\.*\d*)(\w{1,2}).*?(\d+\.*\d*).*/ ) {
				my $val1 = check($1,$2);
				my $val2 = check($3,$4);
				my $val3 = check($5,$6);
				my $val4 = check($7,$8);

                push(@finished_min_request_v2, $val1);
				push(@finished_max_request_v2, $val2);
				push(@finished_mean_request_v2, $val3);
				push(@finished_sd_request_v2, $val4);
				push(@finished_sd_percent_request_v2, $9);
				}
			else {print "";}
		}
	}	


	my $mean_finished_total_time_v2 = sprintf("%.4f", average(@finished_total_time_v2));
	my $mean_finished_req_per_sec_v2 = sprintf("%.2f",average(@finished_req_per_sec_v2));
	my $mean_finished_speed_v2 = sprintf("%.2f",average(@finished_speed_v2));
	my $mean_finished_traffic_total_v2 = sprintf("%.0f",average(@finished_traffic_total_v2));
	my $mean_finished_traffic_header_v2 = sprintf("%.0f",average(@finished_traffic_header_v2));
	my $mean_finished_traffic_data_v2 = sprintf("%.0f",average(@finished_traffic_data_v2));
	my $mean_finished_min_request_v2 = sprintf("%.4f",average(@finished_min_request_v2));
	my $mean_finished_max_request_v2 = sprintf("%.4f",average(@finished_max_request_v2));
	my $mean_finished_mean_request_v2 = sprintf("%.4f",average(@finished_mean_request_v2));
	my $mean_finished_sd_request_v2 = sprintf("%.4f",average(@finished_sd_request_v2));
	my $mean_finished_sd_percent_request_v2 = sprintf("%.2f",average(@finished_sd_percent_request_v2));
	my $mean_finished_succeeded_v2 = sprintf("%.0f",average(@finished_succeeded));
	my $mean_finished_failed_v2 = sprintf("%.0f",average(@finished_failed));
	
	open(my $fh, '>>', $filename) or die "Could not open file '$filename' $!";
    print $fh "$n,$mean_finished_total_time_v2,$mean_finished_req_per_sec_v2,
    			$mean_finished_speed_v2,$mean_finished_traffic_total_v2,$mean_finished_traffic_header_v2,
    			$mean_finished_traffic_data_v2,$mean_finished_min_request_v2,$mean_finished_max_request_v2,
    			$mean_finished_mean_request_v2,$mean_finished_sd_request_v2,$mean_finished_sd_percent_request_v2,
    			$mean_finished_succeeded_v2,$mean_finished_failed_v2\n";
    close $fh;
		
}			

my $end_datestring = gmtime();
open(my $fh, '>>', $filename) or die "Could not open file '$filename' $!";
print $fh "$end_datestring\n";
close $fh;

			
\end{lstlisting}


\end{document}
