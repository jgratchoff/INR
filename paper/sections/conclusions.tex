\section{Conclusions}
\label{conclusion}

The research has identified differences between the HTTP/1.1 and HTTP/2 in terms of response and request time, amount of header data produced, possible request rates and impacts on the server utilization. A full benchmark setup has been developed consisting of several HTTP clients in different geographically locations and a server instance that delivers responses for both HTTP protocol versions. Moreover webpages of different sizes have been created to simulate "real-world" scenarios. In order to have accurate measurement results a benchmarking script \ref{sec:appendices} that makes use of h2load \cite{h2load} was deployed. 
\\
The measurements reflect the performance improvements a client will experience in the near future by using the new HTTP/2 protocol. Those improvements are mainly based on header compression mechanisms and more effective usage of TCP connections. Particulary high latency connections will benefit from the multiplexing mechanism that was introduced in HTTP/2. HTTP/2 will result in significant less TCP connection establishments (TCP three-way handshakes) on server side and thus it will save a lot of costly packet round trip times that would be required to fetch the same web page by using HTTP/1.1.
\\
It is important to consider that all measurements were performed by using clients running on the  Amazon EC2 infrastructure. Thus an assured allocation of the VM resources is not fully guaranteed. That is the drawback of a shared infrastructure and can have impacts on the measurements. Furthermore network pathes towards the webserver can change seen from the Amazon infrastructure or other events can happen that might impact round trip times.  
\\
A significant difference between HTTP/1.1 and HTTP/2 in terms of network load or CPU utilization was not detectable on the server. The main difference that has been discovered on server side is the TCP connection handling. It is faster and more effective when HTTP/2 is used, especially significantly less TCP TIME WAITS sockets were discovered on the server while performing HTTP/2 measurements.
\\
Large web service providers need to consider some technical implementation details when switching to the new protocol. First, so called deep inspection packet filters need to be adjusted in order to let HTTP/2 packets to pass through. Those devices cannot longer inspect HTTP/2 traffic on application level, like it was used to be with encrypted or unencrypted plain text HTTP/1.1 connections. The protocol is now binary and a simple telnet connection to a web server in order to conduct an HTTP GET request manually will not be possible with HTTP/2 anymore. Furhermore "hacks", like Sharding or Sprinting which were introduced in HTTP/1.1 to solve performance issues of the protocol are no longer necessary. In fact those HTTP/1.1 workarounds will lead to performance loss rather than to performance gain, if combined with HTTP/2.    

