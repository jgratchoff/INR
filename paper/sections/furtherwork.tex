\section{Further Work}
\label{furtherwork}

As it is a recent specification new implementations of the protocol will show up. Moreover the big players of server implementations such as NGINX\cite{nginx} and apache \cite{apache2} have not implemented the protocol yet. But this is just a question of the time and probably they will wait until the protocol is specified in an RFC. In the meantime new implementations could be tested in comparison to see which one perform the best. At the time of writing the interesting server implementation of the protocol that are available are nghttp, H20, node-http2, openLiteSpeed. The list of all the implementations are available on the HTTP/2 github page.\cite{http2-imp}
Further work could be done by testing every new features of the protocol in order to show where are the improvement/drawbacks of this protocol in details per feature. 
HTTP/2 is still able to support HTTP/1.1 request with a proxy for legacy purposes. An interesting research subject could be how the  HTTP/1.1 legacy in HTTP/2 can slow down the adoption of the new version of protocol. This has been the case for IPv6 and by showing the improvements in performance for HTTP/2 it would be a pity to see legacy hold improvements in technology back again.