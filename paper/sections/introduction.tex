\section{Introduction}
\label{chap:intro}
HTTP/1.1 is present since 1999 and it is such a big protocol and has been reviewed so many time that the IETF has split the original RFC (2616) into six different ones (7230-7235) describing the whole protocol in details. After more than 15 years of use it was time for a change. All the tricks not to slow the HTTP/1.1 protocol can be forgotten as on February 18th 2015, the specification for the new HTTP protocol HTTP/2 (and HPACK), has been formally approved by the IESG and is on the way to become an RFC standard.  
The main focus during the development period of the successor of HTTP/1.1 was to improve the performance, and thus provide a better user experience. The performance improvement is mainly based on how the packets are sent over the wire within a HTTP/2 session. HTTP/2 data is sent in binary format and is based on a multiplexing mechanism that allows a single connection for parallelism. Concerning security, HTTP/2 will not make the use of TLS mandatory. However, leading browsers firms, Mozilla Firefox and Google Chrome have already mentioned that HTTP/2 will only be implemented over TLS. \\
There are already web client and server implementations that support the final HTTP/2 specification. This research is intending to show what are the impacts of implementing HTTP/2 with TLS compared to HTTPS in a large environment.