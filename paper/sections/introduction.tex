\section{Introduction}
\label{chap:intro}
Specified by the IETF in 1999 in multiple RFCs(7230-7235\cite{RFC723x}), HTTP/1.1 is a standard protocol for web-browsing. HTTP is the essence of the web and most of the users are using this protocol on a daily basis. After more than 15 years of use it was time for a change. All the workarounds not to slow the HTTP/1.1 protocol have to be forgotten as on February the 18th 2015, the specification \cite{http2} for the new HTTP protocol HTTP/2 (and HPACK\cite{hpack}), has been formally approved by the IESG and is on the way to become an RFC standard.   \\
The main focus during the development period of the successor of HTTP/1.1 was to improve the performance, and thus provide a better user experience. The performance improvement is based on how the packets are sent over the wire within a HTTP/2 session. HTTP/2 data is sent in binary format and instead of creating a single TCP session per element retrieved, HTTP/2 use TCP streams that are multiplexed and can be prioritized. The purpose of this technique is to reduce to one the number of TCP sessions created every time a user access a web page. Concerning security, HTTP/2 will not make the use of TLS mandatory. However, leading browsers firms, such as Mozilla Firefox and Google Chrome have already mentioned that HTTP/2 will only be implemented over TLS. Thus ensuring that the data will be encrypted between two end parties. HTTP/2 introduces several other new features that will be presented in this paper. \\
There are already web client and server implementations\cite{http2-imp} that support the final HTTP/2 specification. This research is intending to show what are the impacts of implementing HTTP/2 with TLS compared to HTTPS in a large environment. A benchmark will be performed on a physical web server that will be tested with a high number of virtual clients located at different places around the world to outline the impact of the round trip time (RTT) on both version of the protocol and to see which performs the best. The research will draw conclusions on what impacts does the new protocol can have on large web service providers infrastructure.