\section{Literature review}
\label{relwork}
\subsection{HTTP/1.1 drawbacks}
Specified in multiple RFCs(7230-7235), HTTP/1.1 is a standard protocol for web-browsing that is now used for more than 15 years. This protocol has been reviewed a number of times and contains a high number of options. It uses a client-server model where the clients is most of the time a browser that is getting data from a server located in a different location. HTTP is the essence of the web and most of the users are using this protocol on a daily basis. 
Since the early beginning of the protocol, users complained about the time that takes a page to load. One big reason of this problem was not due to the protocol but to the speed and the reliability of the Internet. However since the web has changed and more and more users have a high and reliable connection to the Internet. This and the growth of clients on the web has led to the increase of web pages size and the number of elements on a web page. HTTP was not designed for this use as indeed it is inadequately using the TCP protocol by retrieving one element from a web page using one TCP connection. It is thus not taking full advantage of the high performance TCP protocol and thus it is leading to a slower loading time. The loading time is also highly dependent on the Round Trip Time (RTT) and some efforts have been done on improving this aspects over the years by adding geographical redundancy however the problem is still present for low cost organisation that are only able to develop server in a single location. 
//SPEAK ABOUT TCP head of line blocking here//
This latency directly affect the clients and can be critical for your website (e.g. users leaving the page as it takes too long to load). So over the years, web developers have tried to reduce this critical factor that is latency. In order to do so the developers have been tried to adapt themselves to the protocol by trying to reduce the latency by doing many tricks or workarounds. The most famous of these workarounds are:
\begin{itemize}
\item Spriting: The fact to create an image containing many pictures destined to be present on the website and let the browsers display (via CSS or Javascript) the picture wanted by cropping/cutting out the single image from the big one.
\item Sharding: In order to overcome the problem of increasing TCP connections per domain, developers started to create several domains, holding different part of the website. This decreases the page loading time by reducing the number of connection per host. Thus leading to a better performance of the HTTP/1.1 protocol.
\item Concatenation: In order to reduce the number of TCP connections, devellopers started to concatenate files (e.g. javascript) into one big files. 
\item Inlining: By embedding the data straight in the CSS in base64 format it avoids to send picture and thus creating new TCP connection. 
\end{itemize}
All this workarounds were needed as the page loading time and the number of requests were increasing so much that the web was slowing down even though more and more people had access to a high and reliable connection. After 15 years, it was time for a change. That is why the IETF created a working group named HTTPbis that started working on a new protocol. Beforehand Google started working on a new protocol, called SPDY, that was aiming to encounters the problems from the HTTP/1.1 protocol. The HTTPbis group started working from a working concept of protocol in the name of SPDY/3 (draft). And this was the start of HTTP/2.

\subsection{HTTP/2 improvements}
This section is subdivided into several ones describing how it improves from its predecessor HTTP/1.1 and emphasis on the reasons to move to the new HTTP/2 protocol for large scale environments. 
\subsubsection{Binary format}
HTTP/1.1 is based on a text/ascii format which is for sure an advantage for humans to read and thus to debug the protocol. It is described by Raymond as "easy for human beings to read, write, and edit without specialized tools"(http://www.catb.org/esr/writings/taoup/html/ch05s01.html). However for clients and server ascii is not their mother tongue. Indeed, without any surprises, computers are using binary as a format for exchange. HTTP/2 is using this format.
Binary is know to be much more efficient for binary structures. It is indeed hard to define the start and the end of a field in text based protocols. However with binary format it is much more natural. Binary will then improve the structure of the protocol and thus the efficiency of the protocol "HTTP/2 also enables more efficient processing of messages through use of binary message framing."(https://datatracker.ietf.org/doc/draft-ietf-httpbis-http2/?include\_text=1)

\subsubsection{Multiplexing and priorities}
\subsubsection{Header compression}
One of the problem of HTTP/1.1 is that when a high number of pages are retrieved from the same server the two will be very similar
\subsubsection{Flow control}
\subsubsection{Server push}
\subsubsection{Security}

\subsection{Related work}
The new HTTP2 protocol has been based on the SPDY protocol developed mainly by Google. As shown by Google \cite{google2x}, the SPDY protocol meets its expectations by reducing the loading time of web pages by 55\%. Other people have tried to look into the protocol and one of the most interesting analysis has been done by Servy\cite{servy}. Servy evaluated the performance of the web servers implementing the SPDY protocol comparing it to HTTP/1.1 and HTTPS. The load testing tool used for this benchmark was the NeoLoad 4.1.2. His results showed that the implementation of SPDY increases by a factor of 6 the number concurrent of users possible before errors start showing up in comparison to HTTP and HTTPS. 
A contradictory study showing some boundaries of implementing SPDY has been done by Podjarny\cite{podiatry}. He shows that most of the websites use different domains and as SPDY works on a per-domain basis it does not necessarily help it to be faster. Finally, Wang et al.\cite{wang} have investigated the performance of SPDY for the improvements of the protocol compared to HTTP/1.1. This study highlights that SPDY is much faster since its benefits from the single TCP connection mechanism. However, they also mention that SPDY degrades under high packet loss compared to HTTP. 
Concerning the new standard HTTP/2 a few benchmarks have been performed by the creators of different client/server platforms. They reach the same conclusion for SPDY. \\
However, studies on comparisons between HTTP/2 (draft-ietf-httpbis-http2-14 \cite{h2c-14} ) and HTTP/1.1 with regards to concurrent clients and increasing amount of requests in different geographical locations and different page sizes or amount of elements a webpage contains, have not been conducted yet.