\subsection{HTTP Server and Clients}
\label{subsec:server_client}
HTTP clients that conduct measurements from long distances are deployed using Amazon t2.micro instances. T2 instances provide a baseline level of CPU performance that is comparable to 2 vCPUs with the ability to burst above the baseline level \cite{amazon-ts}. For local tests within the OS3 network, XEN VM are used. Each virtual machine had 2 vCPUs and 2 Gigabyte RAM assigned. The clients were located in different geographical areas, resulting in different RTT times between clients and server. On each client h2load was compiled and the wrapper script uploaded. Table \ref{table:locations} shows all clients and their corresponding average round trip times to the server.

\begin{table}[h]
	\centering
\begin{tabular}{ | c | c | }

\hline
\textbf{Location} & \textbf{RTT in ms}\\ \hline \hline
Tokyo/Asia &  280\\ \hline
North Carolina/North America &  150\\ \hline 
Frankfurt am Main/Europe &  7\\ \hline
Amsterdam/Europe (local) &  0.3\\

\hline
\end{tabular}
\caption{RTT (in ms) per location}
\label{table:locations}
\end{table}

The server that provides access for the HTTP clients is not virtualized. It has 8 Intel Xeon CPUs (1,87GHz) and 8GB RAM installed. The web server has a public IPv4 address and is connected to the public Internet via a 1 Gbit Ethernet interface. As HTTP/2 server nghttpd \cite{nghttp} version 0.76-DEV is used which listens on TCP port 8881. For HTTP/1.1 requests Apache2 \cite{apache2} in combination with nghttpx \cite{nghttpx} is used. The Apache2 configuration has been adjusted in order to allow the webserver to start a maximum number of 750 server worker processes. This adjustment was required because Apache starts multiple worker processes depending on the amount of requests, and we discovered that the default configuration was not sufficient for our setup. Nghttpx is a reverse proxy and accepts HTTP/2, SPDY and HTTP/1.1 over SSL/TLS on TCP port 8443 via its front-end. The most recent version of nghttpx (0.76-DEV) at the moment that research was done is used. The protocol to the back-end is HTTP/1.1. The usage of a reverse proxy enabled us to use our measurement tools without modification, although native HTTP/1.1 requests instead of using a reverse proxy would probably result in more accurate measurements. Since our time was very limited, we decided to go for the reverse proxy option.

