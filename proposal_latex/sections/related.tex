\section{Related work}
\label{relwork}
The brand new IETF accepted HTTP2 protocol has been based on the SPDY protocol developed mainly by Google. As shown by Google \cite{google2x}, the SPDY protocol is answering its expectation by reducing the loading time of web pages by 55\%. Other people have tried to look into the protocol and one of the most interesting is Herve Servy's post\cite{servy}. Servy evaluated the performance of the web servers implementing the SPDY protocol comparing it to HTTP1.1 and HTTPS. The load testing tool used for this benchmark was the NeoLoad 4.1.2. His results showed that the implementation of SPDY increases by a factor of 6 the number concurrent of users possible before errors start showing up in comparison to HTTP and HTTPS. The fact that SPDY is using a single connection for all requests induce that the clients are using one worker instead of multiple in HTTP and HTTPS. That makes the server able to handle more users with the same amount of worker. Servy also looked into the repercussion of the SPDY implementation in terms of CPU and memory consumption at the server side. Compared to HTTP, SPDY requests consume less memory but more CPU. However compared to HTTPS SPDY requests consume less memory and CPU usage.
A contradictory study showing some boundaries of implementing SPDY has been done by Podjarny\cite{podiatry}, he shows that most of the website uses different domains and as SPDY works on a per-domain basis it does not necessarily help it to be faster. Finally, Wang et al.\cite{wang} have investigated the performance of SPDY for the improvements of the protocol compared to HTTP. This study highlight that SPDY is much faster as benefiting from the single TCP connection mechanism. However they also mention that SPDY degrades under high packet loss compared to HTTP. 
Concerning the new standard HTTP2 a few benchmark have been done by the creator of the different client/server platforms. They all fall down in the same conclusion for SPDY. However no study shows its impact on the infrastructure.